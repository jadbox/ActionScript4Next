% Copyright (c) 2014 Adobe Systems Incorporated. All rights reserved.

% Licensed under the Apache License, Version 2.0 (the "License");
% you may not use this file except in compliance with the License.
% You may obtain a copy of the License at

% http://www.apache.org/licenses/LICENSE-2.0

% Unless required by applicable law or agreed to in writing, software
% distributed under the License is distributed on an "AS IS" BASIS,
% WITHOUT WARRANTIES OR CONDITIONS OF ANY KIND, either express or implied.
% See the License for the specific language governing permissions and
% limitations under the License.

\section{Introduction and Motivation}
The ActionScript\textsuperscript{\textregistered} language has not changed since
2006. At the time of this writing it is still in the same version 3 without any
significant updates.
But the demands on the Adobe\textsuperscript{\textregistered}
Flash\textsuperscript{\textregistered} platform and on its language have
shifted. We now introduce a new version of ActionScript that is oriented at the
requirements of larger and more performance-critical applications across a wider
range of deployment scenarios than anticipated for its predecessor: ActionScript
4.

The Flash platform has a rich ecosystem into which it deploys programs in a
variety of forms: packaged applications with a captive runtime, loaded
applications on a standalone Flash Player or on a full Adobe
AIR\textsuperscript{\textregistered} installation, dynamically loaded
applications in web browser plugins. Providing a true cross-platform developer
experience across these different scenarios as well as the related ubiquity are
primary attractions that give the Flash platform a great competitive edge.

In the following we analyze the prospects of each deployment mode and how it
sets the stage for our language design. Next we derive guidance from major
technological trends and from our focus audience: game developers.
Then we summarize all of the above in high level guidelines for our language
design.

\subsection{Smartphones: Apps, Resource Constraints}
The dominant distribution form of programs on smartphones is ``apps'', and the
availability of Flash Player in mobile browsers is severely limited. Apps are
statically compiled and even statically linked. This means that AS4 needs to be
well-suited for static compilation and cannot rely on dynamic optimizations to
achieve acceptable performance. This requirement is exacerbated by smartphones
having relatively weak compute power. Furthermore, power consumption is a major
concern and code performance improvement is a common prescription for this,
since quicker execution reduces the amount of time any given stretch of code
requires power.

For app deployment, we should continue to offer a form of native extensions that
can employ functionality specific to one native platform in a non-portable way.

\subsection{Tablets: Middle Ground}
Initially, tablets are like a large smartphone without a phone function, or more
precisely like a large iPod Touch. So they emphasize {\em apps} over {\em web}
just like smartphones. But we have to be prepared for {\em web} also. Quite
possibly, tablets will supplant large parts of PC install base, and then this will perpetuate the
requirements we are facing now for desktops.

Already, we are seeing tablets with physical keyboards, which makes them a lot
more similar to laptops. And for people who enjoy virtual keyboards more than
physical ones, one can argue that tablets have been similar to laptops already.
So either way, the result is pretty much equivalent to a laptop with touch
input. And touch input is just arriving on laptops and even regular desktops,
too. In other words, the differences between tablet, laptop, and desktop are
blurring. As a result of this, we will continue to face devices with PC-class
capabilities, in whatever physical shape, and this will remain relevant.

The real question is what artificial restrictions will be imposed on these
platforms regarding runtime deployment, application code deployment, and dynamic
code generation. Recently, we have experienced a trend towards more close
control. Furthermore, Flash Player deployment is also constrained by our own
capacity to conduct and support porting efforts, especially in the relatively fragmented
mobile space, including tablets. However, for all we know, there will continue
to be at least some tablets that admit Flash Player to browsers.
Since it is hard to predict market share and constraints concerning all platform owners
for extended periods of time, we must stay nimble. This means keeping support for
JIT deployments enabled, even for tablets.

On the other hand, there are more and more cheaper, low-end tablets. This
perpetuates the alignment with smart phones. Apps are here to stay as
deployment form as well.

So unsurprisingly, we see tablets positioned in the middle between smartphones
and desktop, requiring both AOT and JIT deployment. Furthermore, the performance
arguments made in reference to smartphones apply typically also to tablets,
though lessened by their relatively larger compute and battery capacity.

\subsection{Desktop: High End Clients}
No matter how successful tablets become, desktops stay relevant. Many uses are
better served by their larger form factor and their superior compute power and
being mobile is not always required. Furthermore, the monetization of software
and services on mobile devices is relatively weak, even though the cross-over
point for mobile device numbers trumping PC numbers has been reached.

By definition, the desktop device class carries the high end for compute clients
(including workstations). There did not use to be a choice between PCs and
mobile devices. Now that there is, the higher investment in a PC implies
indications concerning the user's objectives. Why would anyone choose a PC over
a tablet or smartphone?
\begin{itemize}
  \item Performance
  \item Screen real estate
  \item Memory
  \item Storage and storage bandwidth
  \item Network bandwidth
\end{itemize}

Today, the deployment of Flash Player is almost unrestricted on desktops. The
plugin is virtually ubiquitous. And it requires a JIT to perform well. This
gives us the opportunity to put features into our language that provide the
ultimate performance experience, fully exploiting dynamic optimization.

For JavaScript, and also for ActionScript until now, dynamic compilation is
first of all a compensating measure for inherent performance impediments.
Aiming to eventually make performance a competitive differentiator for
ActionScript, we now prefer to drive performance from a language design position
that lets performance optimizations take off from higher ground.

Today's desktops have such abundance of memory that there is no question that we
have to support 64-bit computing. This has to be taken into account for mass
data types in ActionScript.

\subsection{Servers: Outlook}
For various reasons, the Flash platform is a client-centric technology without
significant deployment on servers. Among those reasons, insufficient performance, the lack
of concurrency, and the resulting poor hardware utilization have to be seen as
major impediments. It is expected that these downsides will be mitigated and
eventually overcome by our efforts on the client side. At least from a language
and VM perspective there should then be no barrier to running (large) AS4
applications on servers. Besides, this also applies to server-side AS4
in cloud computing.

If and when AS4 will be running well on servers, the same arguments that are
made today on desktops wrt. performance, memory, storage, and bandwidth will
apply, only more so.
We can count on 64-bit and multi-core requirements. Furthermore, our language
design should be on a trajectory that can lead to efficient expression of distributed
programs.

\subsection{Multi-Core and Many-Core}
Mobile devices increasingly feature multi-core CPUs and on desktops they are
already common place. With today's single-threaded ActionScript or even with AS
``workers", this means unused hardware, poor utilization, categorically
restricted performance. Furthermore, power efficiency is crucial on mobile
devices and performance is a key driver for power savings. Combining these two
observations, we cannot afford to leave the available hardware concurrency
presented by multi-core CPUs untapped.

With frameworks like Stage3D and Starling, Flash Player is successful in
harnessing the power of GPUs for rendering and making it available in a cross-platform fashion.
But less graphics-oriented compute tasks are underserved. We need to tap more
into the available parallelism on GPUs and GPGPUs to drive performance beyond
multi-core capacity.

We are working towards integrating concurrent/parallel programming seamlessly
into ActionScript. This already affects earlier features of AS4. We are strongly
motivated to facilitate functional programming style. In particular, this
includes suppressing mutability of values where possible, i.e. wherever
tolerable.

\subsection{HTML5: Redividing the Application Space}
HTML, CSS, and JavaScript have become more expressive and performant and capable
of fulfilling roles that used to be covered by Flash Player alone. Many
analysts, bloggers and article commenters claim that the Flash platform will or
at least should be displaced {\em completely}. Yet
there is no compelling \code{technical} reason that reduces the remaining realm of the Flash platform
overwhelmingly, and it can grow without serious challenge in areas where
HTML/JavaScript will never be able to catch up. 
JavaScript programming has at least these limitations:
\begin{itemize}
  \item It scales poorly in program size and in team size.
  \item IDE support is hindered by dynamic typing.
  \item Debugging effort is multiplied by the lack of static checking.
  \item The achievable performance ceiling is limited by dynamic features.
  \item It is ill-suited for static compilation and thus JavaScript apps
  often cannot be competitive with native apps.
  \item It is all too easy to write badly performing code, to fall into
  performance anti patterns.
  \item Innovation progress is throttled by standardization.
  \item Its implementation is fragmented and there are economic incentives for
  its standardization members and for its implementors to fragment it as well
  as to reconcile it. At the very least, we can expect its fragmentation to
  oscillate.
\end{itemize}
Despite all this, HTML5/JavaScript is here to stay. One cannot not serve the
Web! Here it is our task to provide a superior alternative where HTML5/JavaScript is
bound to lack. This amounts to a vast space, not just a niche.
The Flash platform should be able to continue its role as technological trail
blazer.

There is still a large overlap in the application space between HTML5
and the Flash platform and we do not expect it to shrink away entirely. We
should be prepared to continue seeing Flash technologies used for general
purpose web programming, ads, banners and so on. We should therefore keep a certain amount of dynamic features to
interoperate with web technologies and to provide smooth onramping of web programmers to
ActionScript 4. The latter can also be facilitated with gradual
typing and type inference.

However, even though we position Flash technologies as a general purpose
programming platform, including web programming, our focus of attention has
moved on to the following topic.

\subsection{Game Development}

Gaming is the ideal technological focus area for further developing the Flash
platform.
\begin{itemize}
  \item The Flash platform already enjoys a large user base in web and app
  gaming.
  \item This is a fast growing, revenue rich product area.
  \item Adobe as a creative design company must play in this space, since it is
  the pinnacle of creative design, posing the greatest challenges.
  \item The requirements for game programming subsume those of most other disciplines.
        If we master gaming, then the technology developed for it will be
        applicable to virtually everything else.  This can be seen as
        trickle down tech transfer from the most demanding area of all.
  \item Game developers are early adopters, more forgiving wrt. initial problems
  in newly developed offerings than most other communities.
\end{itemize}

Besides media asset creation, game programming can generally be categorized into
these main areas:
\begin{center}
\begin{tabular}{ | l | l | l |}
\hline
{\em Category} & {\em Example Tasks} & {\em Typical Programming Styles} \\
\hline Game Logic & player actions, story line/script, UI, events &
object-oriented, imperative \\
\hline Numeric computation & physics, strategy, constraint
solving & functional, imperative \\
\hline Rendering & 3D/2D shading, sprites, video, images, effects & declarative,
constructive
\\
\hline Communication & network / distributed programming & object-oriented,
reactive
\\
\hline
\end{tabular}
\end{center}
In the short term, and in the scope of this text, we concentrate on the former
two of these.

\subsubsection{Social Gaming}
Social games are booming. The market for them has become very competitive. The
effort going into each individual game development is high. Team sizes often
reach into several dozen developers, not rarely to more than a hundred.
This clearly is programming in the large, which could benefit greatly from
better language and tool support.

Frame rates convert into dollars. Here once again, improving Flash
Player performance is key.

\subsubsection{AAA Gaming}

Some game engines run thousands of different tasks in a quasi-concurrent
fashion. To allow developers to map these directly to threads way beyond the
point of oversubscribing multi-core parallelism, we aim to offer a two-level
thread model in a later version of ActionScript.

We also need to leverage all available hardware for numeric computation,
whereas rendering is being served sufficiently well with API approaches.

ActionScript game code can be mixed with code written in other languages
such as C++ when compiling the latter with FlasCC.

\subsection{Conclusions and Goals}
From the above, we gather these requirements:
\begin{description}
\item[Cross-platform:] Cross-platform development remains the main attractor,
even though the dynamics of device and OS regulations
have shifted our deployment modes in recent year as follows.
\item [Both JIT and AOT:] The range of application deployment targets clearly
indicates that both JIT and AOT are here to stay for the foreseeable future and
that we should stay nimble in that regard anyway. This means that if
AS4 is supposed to be a high performance language, it needs to be shaped so that
great performance is possible without relying on dynamic optimizations.
\item[Maximum performance:] In order not only to catch up with competitors,
but to make performance a defendable differentiator for AS4, we have to design
it so that virtually nothing is in the way of performance maximization.
Especially under the aspect of AOT compilation this means that static typing
has to be the default setting and that it has to be absolutely stringent
wherever possible unless the
developer opts out explicitly (by using the \code{*} type). In general, we favor making
non-dynamic features fast over whatever the consequences for dynamic features
are. We also arrange the type hierarchy so that the source compiler can arrange
autoboxing. Operations on primitive types do not have inherent control
flow (when statically typed). Furthermore we introduce fixed-length
arrays, and we plan for multi-dimensional arrays with VM-determined
data layout.
\item[Machine types:] Primitive data types must translate directly to hardware
operand types to provide maximum performance, but also to facilitate
mapping C data types to ActionScript via FlasCC, and to enable us to
develop a foreign interface. The latter will eventually make native extensions
simpler to write and better performing, and it is intended to greatly simplify
the implementation of the Flash Platform itself. To these ends, we should also
offer user-defined value types that aggregate primitive types.
\item[Hardare utilization:] We carefully design the first version of AS4
for the later introduction of high-performance multi-threading and parallelism features,
since offering these is necessary to provide competitive hardware
utilization.
\item[Programming in the large:] Competitive pressures, in particular in game
development, drive our developers towards larger efforts that require more
coordination and more quality frontloading. This also enforces our choice to
support static strong typing. Furthermore, our core language design must be
suitable for the introduction of generics (parametric polymorphic typing) in the
future. This will enhance expressiveness while maintaining static typing.
\item[Both object-oriented and functional:] Large program organization
benefits from object-oriented style and a variety of well-known related
patterns. Yet we want to promote functional programming style for
more algorithmic program parts, especially when these involve concurrency.
Reconciling the two styles, we aim at making object-oriented program parts more amenable to functional programing.
Concretely this dictates giving immutability preference over mutability
whenever possible. To this end we introduce two-phase constructors, which
facilitate declaring and guaranteeing truly constant object fields, virtually
without any loss of generality wrt. what constructors can express. Besides this
feature, we are generally promoting immutability throughout the language to make
developers more accustomed to paying attention to it. For example, function parameters are by default
immutable in AS4.
\item[Small and large memory:] We can future-proof certain mass data types like
arrays and strings by allowing them to range over 64 bits while at the same time
allowing the VM to choose a space-efficient representation on constrained
devices. However, we should also offer the usual 32-bit variants, since small
arrays and strings are common.
\item[Supplemental dynamic features:] Every static type system has its limits
wrt. expressiveness. We should supplement it with select dynamic features and a
comprehensive reflection API. But we must be careful to not compromise our other
goals by doing so.
\item[Type inference:] Static typing does not always require stating types explicitly. The most
suitable types that one might have written can more often than not be discovered
by the compiler and inserted automatically. This mitigates the typing effort and
visual impact caused by static typing compared to dynamic typing. However, type
inference can also be useful even in dynamically typed code contexts, providing opportunistic
performance improvements.
\end{description}

Adobe is not only going to update the ActionScript language but also going
to renovate the libraries in Flash Player, moving from V11 versions to V12. In
both cases, there will be a compatibility break that affects the source code level as well as the binary level.
AS3 will not run on V12 and AS4 will not run on V11. AS4 and V12 will be a fresh new reinvention of the Flash
platform, not just an incremental upgrade.\footnote{AS3 and V11 content will continue to run in Flash Player, which will dynamically
detect which runtime version to launch.}

However, both the AS4 language and the V12 APIs will look very similar and familiar to existing developers.
This is of course intentional, to facilitate migration to our new platform version. Yet we believe that given its
more concise and disciplined design, AS4 will be easier to learn than AS3 for many of those who will come to our
platform as new developers.


\subsection{Overview}
To fulfil all of the above goals, it became necessary to drop some features from AS3. These are
discussed in the next section. Other features, while kept, have been modified as explained in
section~\ref{changes}. In principle, significant portions of AS3 code can be ported to AS4 given this knowledge alone.
But we are aware that this is temporarily limited by some valued AS3 features having been removed without replacement.
Eventually, there will be powerful substitutes that fit in well with the new language.

The following sections present new features in AS4. We first describe those designated for an initial release
(section~\ref{new}), then we sketch the next wave (section~\ref{upcoming}). Finally, we give an outlook on features
(section~\ref{future}) that are further out in our schedule. This is to provide a first impression of the overall
direction and significance of the AS4/V12 platform.

The appendix contains implementation notes (section~\ref{impl}) and a listing of base classes and interfaces
(\ref{base}), which includes core classes in the type hierarchy as well as a new reflection API.

In the remainder of this text we assume that the reader is already familiar with
ActionScript 3.
