% Copyright (c) 2014 Adobe Systems Incorporated. All rights reserved.

% Licensed under the Apache License, Version 2.0 (the "License");
% you may not use this file except in compliance with the License.
% You may obtain a copy of the License at

% http://www.apache.org/licenses/LICENSE-2.0

% Unless required by applicable law or agreed to in writing, software
% distributed under the License is distributed on an "AS IS" BASIS,
% WITHOUT WARRANTIES OR CONDITIONS OF ANY KIND, either express or implied.
% See the License for the specific language governing permissions and
% limitations under the License.

\section{Upcoming Features}
\label{upcoming}
So far, we have just described the intended feature set of the first version of
AS4. In this section we present additional features that we want to introduce as
quickly as possible thereafter.

Note that we cannot guarantee that these or any other features will appear in
any particular form or order. Everything we have written from here on is to be
considered as somewhat speculative.

\subsection{The \code{Symbol} Class for String Interning}
Instances of \code{String} objects are not interned, neither explicitly nor
implicitly. If interning behavior is expected, the \code{Symbol} class must be
used.

\begin{verbatim}
public restricted class Symbol extends String {
    /**
     * "Intern" a given string.
     *
     * 1. If the identical string is already in the global intenring table,
     *    return it.
     * 2. Otherwise, if a string of equal content is already in the table,
     *    return it.
     * 3. Otherwise, enter a copy of the string typed as \code{Symbol}
     *    into the table, then return this resident unique representative
     *    of all strings of equal content.
     *
     * The cost of the initial copying can quickly be amortized
     * when comparing symbols to each other by identity
     * instead of by equality as with non-interned strings.
     *
     * We assume that string interning is predominantly going to be used
     * for relatively short strings, which mitigates the initial interning cost.
     *
     * Explicit coercion of a value to \code{Symbol} (\code{value as Symbol})
     * translates to a call to this method.
     * If necessary, this is preceded by implicit coercion to \code{String}
     * (\code{value as String as Symbol}).
     */
    public static final function fromString(string :String) :Symbol;

    override public function equal(other :*):bool
    {
        return this.identical(other);
    }
    ...
}
\end{verbatim}
All string literals are already \code{Symbol} instances. Additional
\code{Symbol} instances can be produced from Strings by explicit coercion,
as shown below or by direct calls to the above constructor.

\begin{verbatim}
function world():Symbol { return "World"; }

let s1 :Symbol = "Hello"; // OK, because string literals are Symbols already
let s2 :Symbol = "Hello" + world(); // Error, concatenating computed symbols produces a String

let string1 :String = "Hello" + world();
let string2 :String = "Hello" + world();
print(string1 == string2); // true
print(string1 === string2); // false

let symbol1 :Symbol = string1 as Symbol;
let symbol2 :Symbol = string2 as Symbol;
print(symbol1 == symbol2); // true
print(symbol1 === symbol2); // true
\end{verbatim}

\code{Symbol} extends \code{String}, but \code{String} is otherwise not
extensible by user code without certain restrictions, which we explain in the
following section.

\subsection{Restricted Classes}
Generalizing the concept behind class \code{Symbol}, we introduce the keyword
\code{restricted} as class attribute. A {\em restricted} class can be extended
by subclasses, but those must not declare any additional instance fields.
Furthermore, every subclass of a restricted class must be a restricted class,
too.

Class \code{String} is a restricted class and class \code{Symbol} extends it
(see above).

Restricted classes can be used for encoding relevant invariants, which can thus
be statically guaranteed by the type checker. That a string is guaranteed to be
interned when typed as a \code{Symbol} is just one example. Other conceivable
(user-defined) examples are: strings that comply with a certain lexical grammar
(e.g. ``well-formed'' identifiers), lists that are assumed to be sorted,
acyclical graphs, etc. Of course, the user has to be careful that the guaranteed
invariants actually hold and cannot be broken by mutation of constituent
variables.

\subsection{Enumeration Types}
In AS3, finite sets of discrete, enumerated values are customarily represented
by string constants, which is a poor choice for various obvious reasons.
Alternatively, one can use integer numbers, but this is not ideal either. There
is a pattern that provides a more type-safe and comfortable encoding, but it
requires a lot of boiler plate code and strict adherence to certain protocol.
Example:
\begin{verbatim}
public final class RbgColor // AS3 enum pattern code:
{
    private var _ordinal:int;

    public function get ordinal():String
    {
        return _ordinal;
    }

    private var _name:String;

    public function get name():String
    {
        return _name;
    }

    public function RgbColor(ordinal:int, name:String)
    {
        _ordinal = ordinal;
        _name = name
    }

    public static const RED:RgbColor = new RgbColor("red");
    public static const GREEN:RgbColor = new RgbColor("green");
    public static const BLUE:RgbColor = new RgbColor("blue");
}

var myColor:RbgColor = someCondition() ? RgbColor.BLUE : someOtherColor();
...
switch (myColor.ordinal)
{
    case RED:
        ...
        break;
    case GREEN:
        ...
        break;
    default:
        trace(myColor.name);
        break;
}
\end{verbatim}
In AS4, this will soon be condensed as follows:
\begin{verbatim}
public enum Color { // AS4 code:
    RED, GREEN, BLUE
}

var myColor :RbgColor = someCondition() ? RgbColor.BLUE : someOtherColor();
...
switch (myColor) {
    case RED:
        ...
        break;
    case GREEN:
        ...
        break;
    default:
        trace(myColor);
        break;
}
\end{verbatim}

There are more elaborate patterns for enum constants that have extra
payload properties, and there are equally powerful syntactic enhancements in
languages such as Haxe, C\# , Java, and others. We will take further inspiration
from these and later extend AS4's enum capabilities accordingly as an extension
of the above.

\subsection{Abstract Classes and Methods}
The new keyword \code{abstract} is used as an attribute for both classes and
methods. An {\em abstract} class cannot be instantiated directly. Only its
non-abstract subclasses can.

An {\em abstract} method must not have a function body and it must not be
native. The first non-abstract subclass of its declaring class must override it
with a full method.

Constructor functions cannot be abstract.

A class that contains any abstract method must be declared as an abstract class.

\subsection{Constructor, Method, and Function Overloading}
AS4 will allow several different constructors with different parameter lists in
the same class. And then, more generally, methods or functions with different
function signatures can also have the same name, under certain conditions (to be specified later). This is also known
as ad-hoc polymorphism, in that the compiler disambiguates direct calls to such
methods or functions by discerning function signatures at compile time.

\subsection{ActionScript Workers}
We intend to reintroduce workers and to allow them in every deployement mode in
AS4, not just in JIT-based deployments as in AS3. It is likely that we will
update the Worker API to better leverage AS4 language features.

\section{Future Features}
\label{future}
Last but not least, let us briefly look further ahead towards features that we
have not fully designed yet, but that we are particularly keen on introducing to AS4.
\begin{description}
\item[Immutable Data Structure Types]
\item[User-Defined Value Types]
\item[Generics (Parameteric Polymorphism)]
\item[Collection Library]
\item[Module System]
\item[Multi-Threading]
\item[Nested Transactions on Versioned Data Structures]
\item[Data Parallelism]
\end{description}
We will elaborate on these once we have completed designs in hand.
