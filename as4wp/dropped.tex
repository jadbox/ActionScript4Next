% Copyright (c) 2014 Adobe Systems Incorporated. All rights reserved.

% Licensed under the Apache License, Version 2.0 (the "License");
% you may not use this file except in compliance with the License.
% You may obtain a copy of the License at

% http://www.apache.org/licenses/LICENSE-2.0

% Unless required by applicable law or agreed to in writing, software
% distributed under the License is distributed on an "AS IS" BASIS,
% WITHOUT WARRANTIES OR CONDITIONS OF ANY KIND, either express or implied.
% See the License for the specific language governing permissions and
% limitations under the License.

\section{Dropped Features Relative to ActionScript 3}
\label{dropped}
In this section we list features of AS3 that are being
dropped in AS4, motivate why they are being dropped, and suggest
ways of porting AS3 code that relies on those features to AS4.

\subsection{The \code{with} Construct}

AS3 supports \code{with} statements \`a la ECMAScript 3. Such
statements introduce dynamic scoping in the language, as the following example shows.

\begin{minipage}{\linewidth}
\begin{verbatim}
function f(o:Object) {
    var x:int = 6; // declares local variable x
    var y:int = 42;
    with(o) {
        x = 7; // updates o.x instead of x
        y = 0;
    }
    print(x, o.x); // 6, 7
    print(y); // 0
}
f({x : 6}); // passes an object with property x
\end{verbatim}
\end{minipage}

Dynamic scoping has several drawbacks:

\begin{itemize}

\item Dynamic scoping is expensive to implement, since it requires
maintaining a chain of run-time objects (a.k.a. scope chain) for
resolving lexical references.

In particular, dynamic scoping makes function closures (run-time representations of functions that are passed by
value) heavyweight, since a function closure must capture the scope chain that is in effect
at the point of function definition.

But closures are often used to pass handlers in
event-based programming.

Furthermore, going forward, we want to encourage people to use closures
(as part of a move to parallel programming using maps, for example).

Finally, lightweight closures are required to implement our new object
initialization protocol (see section~\ref{construct2}).

\item Dynamic scoping also causes the compiler to ignore static errors, thereby
requiring dynamic checking to guard assignments to variables, to maintain type
safety, and therefore security, at run-time.
In other words, if we cannot ensure that a variable of type $T$ will always hold
values of type $T$, then the dynamic compiler / VM cannot safely allocate space
for the variable based on $T$ without guarding assignments to that variable;
such guards potentially degrade performance.

\end{itemize}

It is instructive to note that for similar reasons, EcmaScript 5 strict mode
makes it a syntax error to use \code{with} statements.

The main benefit of dropping \code{with} statements is that we recover
lexical scoping, which enables compile-time optimizations.

\begin{itemize}
\item Lexical scoping is efficiently implementable by relying on a flat, compile-time map over
variables (where variables declared in inner scopes shadow variables
declared in outer scopes). This considerably simplifies and improves
the performance of name lookup.

\item Lexical scoping allows the compiler to eliminate dynamic checking on
assignment to variables in scope. This is a big performance
win. In fact, it is unfortunate that statically typed AS3 programs
do not enjoy this benefit, since this performance win is part of the promise of
static typing in general. Indeed the primary goal of static typing is
proving, at compile-time, that dynamic typechecking is redundant:
statically typed programs do not violate type safety.

\end{itemize}

A slow, but mostly functional hack for porting code that relies on
\code{with} statements is as follows. Code inside a
\code{with} block for object \code{o} can be transformed so that
every lexical reference \code{x} is rewritten to a conditional that checks
whether \code{x} can be found in \code{o} and branches to either
\code{o.x} or \code{x}. Essentially, this amounts to inlining the
lookup logic inside \code{with} statements.

Note that the keyword \code{with} is no longer required, although we may want to
reserve it for future repurposing.


\subsection{Namespaces and E4X}

In AS3, namespaces enable a form of dynamic overloading, where different functions
with the same name may be defined in different namespaces and called
by supplying the namespaces at run time. Namespaces were introduced as
part of the EcmaScript 4 effort, which was eventually abandoned (so
that namespaces never made it into other JavaScript implementations).

Namespaces unify a lot of concepts in AS3: packages, access
control modifiers, API versioning. However, arguably all of those
concepts can be readily implemented without namespaces. At the same time, there do not seem to be existing
applications that exploit the full expressive power of namespaces: for
example, namespaces could provide a powerful mechanism to do
capability-based programming, but no such framework seems to be in use by
the AS3 developer community. As such, namespaces seem like ``a hammer
looking for a nail'' that everybody pays for but only few, if at all, use.

The design and implementation of namespaces in AS3 suffer from many problems:

\begin{itemize}

\item When combined
with static binding of function calls to function definitions based on
static types, namespaces have several unintuitive implications, as the
following example shows.


\begin{verbatim}
namespace N1; namespace N2;
class A {
  N1 function f() { print("in A"); }
}
class B extends A {
  N2 function f() { print("in B"); }
}

use namespace N1; use namespace N2;
// adds {N1,N2} to the set of possible qualifiers for every lexical reference

var b:B = new B();
b.N2::f(); // in B
b.f(); // in B
// desugars to b.{N1,N2}::f(), which early binds to b.N2::f() because b:B

var a:A = b;
a.N1::f(); // in A
a.f(); // in A
// desugars to a.{N1,N2}::f(), which early binds to a.N1::f() because a:A

var c:* = ... ? a : b;
c.f(); // in B
// desugars to c.{N1,N2}::f(), which late binds to c.N2::f() because c is a B-object
// this means that we cannot infer c:A
\end{verbatim}

Thus, namespaces preclude type inference for optimization:
any precision lost when doing type inference may cause semantic
unsoundness, i.e., may change the behavior of the
program. Unfortunately, in an object-oriented language, subtyping
is a key characteristic, so losing precision via subtyping is common and in fact
desirable: it is not reasonable to require tracking exact types.

\item Furthermore, namespaces make references heavyweight: every reference consists of
not just an identifier but, in the worst case, a set of namespaces
that need to be resolved at run time; such references may be ambiguous
(i.e., they may resolve to different definitions), and throwing errors
when they are ambiguous requires \emph{all} definitions that possibly
match to be
considered during name resolution (as opposed to just finding
\emph{some} definition that matches).

\item Finally, it is painful to specify, understand, and implement
  namespaces. For example, it is tedious to work out what happens when
  namespaces are themselves defined in namespaces, which are in turn ``opened''
  in some order (via \code{use namespace'}; see the AS3 language specification
  for details on this and other issues. Not surprisingly, the AS3 compiler has
  numerous bugs around namespaces. Furthermore, the VM has several notions of
  names, including multinames and run-time qualified names, that add subcases to
  several core bytecodes; see the AVM2 overview for details. (In fact, it is a
  fair claim that almost nobody inside Adobe enjoys a full understanding of the
  internals of namespaces, as evident from numerous misunderstandings that keep
  coming up in various discussions on the topic: a situation that is untenable
  going forward.)
\end{itemize}

The main benefits of dropping namespaces are:

\begin{itemize}
\item huge reduction in complexity in the language specification and
  the VM implementation;
\item performance improvements (both in time and in space) due to
  streamlining of
  names and the process of name lookup;
\item soundness of type inference, i.e., the run-time
  semantics of the language does not depend on the precision of
  compile-time type information (which is crucial in a language with
  dynamic typing).
\end{itemize}

As long as a function defined in a namespace is always referenced by
explicitly mentioning the namespace (or, all references to it are statically
bindable), the namespace can be simulated by
an extra parameter taken by the function and a corresponding extra
argument passed to every call of the function.

Note that the keywords \code{namespace} and \code{use} become redundant, as do
the operator \code{::}.

In AS3, packages are encoded with namespaces (with package
``imports'' encoded by namespace ``uses''), and access controls such as
\code{public}, \code{protected}, and \code{private} are also encoded with
namespaces. In AS4, packages and access controls are primitive
concepts, as in Java and C\#.

See section~\ref{sec:packages-access-controls} for more detail on packages and access control.

Configuration constants, which dictate conditional compilation, have
thus far been defined in a special namespace \code{CONFIG}. The AS4 syntax
for conditional compilation is explained in section~\ref{sec:conditional-compilation}.

%To aid porting, we can provide XML libraries that existing code can be
% rewritten to use.

Without namespaces, E4X cannot stand on its own, so we remove
it. Consequently, we also remove the associated \code{XML}, \code{XMLList}, and
\code{QNames} classes. Removing E4X also means
removing a lot of cruft from the syntax (see section~\ref{redundantOperators}).

\subsection{Global Object and Package-Level Variables and Functions}
\label{GlobalObject}
There is no notion of a global object in AS4, nor are there any package-level
variables or functions. Instead, there is a {\em main class}, which is anonymous
and is expandable. It is also prepopulated with a few convenience methods such as \code{trace()}.

See section~\ref{programStructure} about program structure for further
details.

\subsection{Functions That Implicitly Bind \code{this}}

In AS3, all functions, including non-method functions (e.g., anonymous
functions, nested functions) implicitly take an additional \code{this} parameter.

\begin{itemize}
\item Thus, all functions can be used as methods, by attaching them
  (explicitly or implicitly) to objects. The
calling convention is such that all calls pass an additional \code{this}
argument, which means that non-method functions are needlessly expensive,
even though such functions are expected to become more common as we
encounter more parallel-friendly code. Also, when there is no explicit \code{this}
argument, currently the global object is
implicitly passed, so without a
global object (see above) we are left with an incomplete feature.

\begin{verbatim}
function F() { this.x = 0; }
var o:Object = { f: F }; // this in F binds to o
o.f();
print(o.x); // 0
o.x = 1;
{ f: F}.f(); // this in F binds to new object
print(o.x); // 1
\end{verbatim}

\item Furthermore, all functions can be used as constructors for new
objects, but without prototypes (see below), the utility of this
feature is greatly diminished.

\begin{verbatim}
function F() { this.x = 0; }
var o:Object = new F(); // this in F binds to o
print(o.x); // 0
\end{verbatim}

\end{itemize}

In AS4, the keyword \code{this} may appear only in an instance method,
and may be arbitrarily nested inside other functions. Functions that
are not instance methods do not bind \code{this}, although they may
have \code{this} in scope when they are nested inside instance
methods.

In AS4, the keyword \code{new} must be followed by a class name
resolved at compile-time.

Furthermore, functions as constructors do not serve as ``cast''
operators. The cast operator in AS4 is \code{as}, whose semantics is
modified to throw an error upon cast failure (rather than returning
\code{null}, which is not type-safe since \code{null} is not a valid
value of all types, in particular value types). The \code{as} operator
must be followed by a type, which could be a value type or a class
name.

Complementing \code{as} is the operator \code{is}, which must also be
followed by a type. AS4 maintains the invariant that \code{v is T} if
and only if \code{v as T === v}, for any compile-time type \code{T}
and run-time value {v}.

In a later AS4 version, we may provide instance methods of
the \code{Class} class corresponding to \code{new}, \code{is}, and \code{as}, so
that such operations can be applied to types computed at run-time.

The main benefits of dropping \code{this}-bindable functions is that
closures and the calling convention becomes more lightweight, and
thus, more efficient.

Functions that implicitly take \code{this} as a parameter can be
simulated by methods in corresponding hidden classes.

\subsection{The \code{undefined} Value}

AS3 supports \code{undefined} as the only value of type
\code{void}; it parallels \code{null}, which is a valid value of type
\code{Object}. The type \code{*} includes \code{void} and
\code{Object}; thus both \code{undefined} and \code{null} are valid values of type \code{*}.

The value \code{undefined} arises in the following cases in AS3:

\begin{itemize}

\item Whenever a dynamic property is not found in AS3, \code{undefined} is returned.

In AS4, missing dynamic properties are denoted by \code{null} instead.

\begin{verbatim}
var o:* = { };
print(o.x); // undefined in AS3, null in AS4
\end{verbatim}

\item In AS3, arrays can have ``holes,'' which are populated by
  \code{undefined}. Holes cannot exist in AS4 arrays.

\item Whenever a function of return type \code{void} is called in AS3,
  \code{undefined} is returned as the result.

In AS4, we throw a compile-time error if it is clear at compile time that a
value of type \code{void} is being expected (because there is no such
value anymore); on the other hand, if it is not clear at compile time
that such a thing may happen, but
it still happens at run time, we throw a run-time error.

\begin{verbatim}
function f():void { }
var r:* = f(); // OK in AS3, compile-time error in AS4
print(r); // undefined

var g:* = f;
print(g()); // OK in AS3, run-time error in AS4

function h():* { return; } // OK in AS3, compile-time error in AS4
\end{verbatim}

As a corollary of compile-time enforcement, return statements are linked to the type signature
of the enclosing function: if the return type is the dynamic type,
then having a \code{return} statement without an expression is a
compile-time error.

As an elaboration of run-time enforcement, if a function is called through the
dynamic type, then the compiler makes it explicit in the emitted
bytecode for the (slow) function call whether it expects a result or
not (e.g., emitting \code{call} vs. \code{callvoid}); and as part of the
implementation of the bytecode, the function signature is looked up to
throw a run-time error whenever a value of type \code{void} is
expected.

\end{itemize}

The main benefits of dropping \code{undefined} are reduced complexity
in the VM, and a uniform interpretation of \code{*} as a boxed
representation for value or reference types without additional conversion.

By closing all loopholes for creating uninitialized variables, there is
no reason left to keep \code{undefined} in addition to \code{null}. In AS4,
variables that are not explicitly initialized and are of type \code{Object} or
one of its subtypes are initialized to \code{null}. All variables of
non-\code{Object} types are initialized to a zero-like default value. The
remaining possibility is a variable of type \code{*}. By defining its default
value to be \code{null}, there is no room left for \code{undefined}. It would
just act as a ``second kind of \code{null}'', which is superfluous and just
complicates the language.


Note that without the \code{undefined} value, we do not have the unary
\code{void} operator.

\subsection{Prototypes}
\label{prototypes}
Prototypes provide yet another inheritance mechanism for objects (parallel to the usual inheritance mechanism provided by classes), as
the following example shows:

\begin{verbatim}
class A {
  var x:int = 6;
}
class B extends A {}

B.prototype.x = 42;
B.prototype.y = 7;
A.prototype.y = 42;
A.prototype.z = 0;

var o:* = new B();
print(o.x,o.y,o.z); // 6, 7
\end{verbatim}

Prototypes have several drawbacks when they coexist with classes.

\begin{itemize}
\item To support prototypes, every object in the language carries an
extra reference to another object, irrespective of whether it is an
instance of a dynamic class or not.

\item Furthermore, the fact that some in-built methods of \code{Object} are
dynamic properties accessed via prototype inheritance rather than class inheritance
has several unintuitive implications.

\begin{verbatim}
class A {
  // shadow, not override! (trait properties shadow dynamic properties)
  public function toString() { return "some A"; }
}
class B extends A {
  // override! (trait properties override trait properties)
  override public function toString() { return "some B"; }
}
class C {
  // not shadow! (dynamic properties are public)
  function toString() { return "some C"; }
}
class D {
  // run-time error! (dynamic properties are ignored by compile-time checks)
  public var toString = null;
}
print(new A()); // some A
print(new B()); // some B
print(new C(), new C().toString()); // [object C], someC
print(new D()); // error!
\end{verbatim}

\end{itemize}

As long as prototype chains are fixed, prototype inheritance can be
simulated by class inheritance.

By dropping prototypes, objects become smaller, and in-built methods of
Object are declared as trait properties in the class itself.

Note that the keyword \code{instanceof} is dropped.

% Early binding of methods on Object implies that names of Object
% instance methods become unusable as dynamic property names.
% { toString: true}

\subsection{Dynamic Classes}
\label{dynamicClasses}
In AS3, every non-final static class can be extended by a dynamic class, which
features dynamic properties. This approximates JavaScript, but falls short of
modelling its entire generality, because dynamic properties in AS3 do not shadow
static properties nor can they replace them nor can static properties be
modified or removed dynamically.

In order to make the class system fit into the prototype system, AS3 models all
functions as methods so that they can be assigned as values to properties. This
means that every function has an implicit \code{this} parameter, which
dynamically binds to the receiver object upon method invocation. And in case of
a function invocation it dynamically binds to the global object.

In AS4 we cannot continue this scheme, because it is inefficient to have an
extra parameter for every function call and because there is no such thing as a
global object any more.

\subsection{\code{ApplicationDomain}}

In AS3, global definitions are organized into a hierarchy of application domains
at run time, with complicated lookup and binding rules. This feature is useful
in combination with dynamic loading.

Initially, both dynamic loading and application domains are absent in AS4.
However, we plan to introduce these capabilities into AS4 later,
after moderate redesign relative to AS3.

\subsection{The \code{Array} Class}

We eliminate the AS3 class \code{Array}. The class name is reused for fixed-length arrays (see
section~\ref{fixed-length-arrays} and appendix~\ref{Array}), which only retain a small subset of the functionality of
AS3 arrays.

\subsection{The \code{Proxy} Class}

We eliminate the \code{Proxy} class in AS4, mainly because its only
existing uses seem to be in Flex. We will re-design \code{Proxy} in
a later version of AS4.


\subsection{The \code{include} Construct}

The \code{include} feature in AS3 allows fragments of code to be
spliced into larger fragments of code. This feature is
problematic for tooling; furthermore, the AS3 parser has several bugs around
this feature. Thus, we drop it in AS4.

\subsection{Automatic Semicolon Insertion}

The optionality of semicolons causes complexity in the language definition and
the lexer/parser. The language specification needs to be extremely pedantic
about where a line break can occur or not to disambiguate what is meant (where
the presence or absence of a required semicolon would immediately disambiguate).
Overall, this feature in AS3 breeds implementation errors as well as, in some
cases, program understanding errors.

\begin{verbatim}
var x = foo // ;
(y as T).bar();
\end{verbatim}
Without the semicolon, the above code could be mistaken for:
\begin{verbatim}
var x = foo(y as T).bar();
\end{verbatim}


\subsection{Assignment Expressions}

In AS4 assignments, e.g., \code{x = e}, \code{x += e},
\code{o.x = e}, \code{o.x += e}, are statements, not
expressions. This means that they cannot appear inside
conditions, argument lists, and so on.

This restriction avoids common programming errors.
\begin{verbatim}
if (x = y) { ... }
\end{verbatim}
We are aware that this problem is greatly mitigated by the fact that there are no implicit conversions between values of
type \code{bool} and numeric types. But we also believe that avoiding inline assignments in general fosters program
readability and understanding, especially when programming in the large in teams.

Last but not least, the above restriction paves the way for named parameter passing in a future version of AS4.
\begin{verbatim}
function foo(x:int,y:String) { ... }
foo(y = "..", x = 5);
\end{verbatim}

\subsection{Rest Parameters}

We eliminate rest parameters in AS4. Calling a function with rest
parameters involves hidden costs that are not apparent to the
programmer. In particular, any rest arguments need to be boxed and packed into
an array. We believe that the programmer is better served when the full
cost is explicit at the call site, i.e., the function takes an explicit
optional array parameter instead.

\subsection{List of Dropped Operators}
\label{redundantOperators}
The following AS3 operators are redundant in AS4, and are therefore dropped.

\begin{description}
\item [unsigned right shift \code{>>>}] -
    cf. overloading of right shift, see section~\ref{sec:operators}

\item [\code{void}] -
    cf. dropped feature: undefined

\item  [\code{instanceof}] -
    cf. dropped feature: prototypes; operator \code{is} can be used to determine
    whether a value is an instance of a given class

\item [\code{typeof}] -
    to be replaced by a method in a reflection package

\item [\code{delete}] -
    it is no longer possible to remove object properties

\item [\code{in}] -
    the structural composition of objects can no longer be inspected without using the dedicated reflection API (see
    appendix~\ref{reflection})

\item [type casting operator \code{T(e)}] -
    replaced by the operator \code{e as T}

\item [attribute operator \code{@}] -
    repurposed for metadata, cf. dropped feature: E4X

\item [descendant operator \code{..}] -
    cf. dropped feature: E4X

\item [namespace qualifier operator \code{::}] -
    cf. dropped feature: E4X

\item [the rest parameter indicator \code{...}] -
    cf. dropped feature: rest parameters
\end{description}
